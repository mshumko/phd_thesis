\chapter{Conclusions and Future Work}\label{conclusions}
In this dissertation we have explored the microburst scattering mechanism directly in Chapter \ref{CH:mageis_microburst} and indirectly in Chapters \ref{CH:bouncing_packet} and \ref{CH:ac6_study}. In Chapter \ref{CH:mageis_microburst} we used particle and wave instruments on the Van Allen Probes to study microburst signatures near their scattering region inside the radiation belts. To these observations we applied the relativistic theory of wave-particle resonant diffusion and found that the motion of the microburst electrons was not along single-wave characteristic curves in momentum phase space, given the spacecraft position, orientation and the plasma environment. This result at first appears to contradict the belief that many members of the community hold, that microburst precipitation is due to a diffusive process. In reality both are probably valid on different time scales. Individual microbursts are probably not scattered diffusively, but the combined contribution of an ensemble of microbursts will have properties that are well modeled as a diffusion process.

The microburst sizes estimated in prior literature as well as Chapters \ref{CH:bouncing_packet} and \ref{CH:ac6_study} show that there is a large variability in microburst sizes although microbursts are generally small. The study in Ch. \ref{CH:bouncing_packet} gave us a glimpse into the dynamics of a rarely observed bouncing packet microburst from a dual point measurement platform. This study has shed light on the lower bound latitudinal and longitudinal sizes of that microburst, and it was found to be larger than microburst sizes reported in recent literature, and somewhat smaller than the microburst sizes observed with high altitude balloons in the mid 1960s. A comparison between satellites separated in latitude and balloons separated in longitude is somewhat an apples to oranges comparison because the microburst shape is still unknown.

The AC6 microburst study in Ch. \ref{CH:ac6_study} showed that in LEO, $60 \%$ of the $662$ microbursts were observed while the AC6 separation was less than a few tens of km while a minority of microbursts were observed up to $\approx 100$ km separation. These conclusions agree with prior literature from high altitude balloons and LEO spacecraft, although as mentioned before the microburst shape makes comparisons somewhat ambiguous. The equatorial microburst size distribution is heavily dominated by microbursts smaller than $200$ km. This is a very small size, highlighting that the waves that scatter microburst electrons must have correlated properties on those scales. A preliminary comparison between the equatorial distributions of microbursts and lower band whistler mode chorus waves shows a better agreement between high amplitude chorus waves and microbursts, although a more thorough study is necessary to address the various systematic biases.

\section{Future work}
An extension of the case study in Chapter \ref{CH:mageis_microburst} is a statistical study using the Van Allen Probes. Other microburst-like events have already been identified by eye. These other events were also simultaneously observed with enhanced wave activity, hence they may be related and a further investigation is warranted. A microburst detection scheme similar to the one used in Chapter \ref{CH:ac6_study} can be easily implemented to automatically identify other microbursts for further study. A few compelling questions that can be addressed with this study are: what is the typical pitch angle extent of microbursts? Do these microbursts have a similar MLT distribution to microbursts observed in LEO? What fraction of microbursts were observed during enhanced wave activity? What wave modes and properties are observed during these events? And lastly, what fraction of microbursts can be modeled with a diffusive process?

Another study related to the electron bounce period analysis done in Ch. \ref{CH:bouncing_packet} can be used to verify magnetic field models and in particular the length of magnetic field lines. Current magnetospheric magnetic field models assume that Earth's internal magnetic field is relatively static e.g. the International Geomagnetic Reference Field, and superpose that field with a highly dynamic field model who's dynamics are driven by the plasma environment in the magnetosphere and the solar wind. The difficulty lies in accurately modeling this dynamic field, and verifying these models is somewhat difficult. One verification technique involves identifying bouncing packet microbursts observed by SAMPEX and FIREBIRD, and then estimate the electron bounce period in a similar fashion to the analysis in  Ch. \ref{CH:bouncing_packet}. The empirical bounce period can then be compared to modeled bounce periods from a variety of magnetic field models, and then model accuracy estimated. Identifying the bouncing packet microbursts is not easy, but may be possible with an auto-correlation or machine learning approaches e.g. a neural network.

The last project described here that can be done with existing data is to test the hypothesis that curtains, which were briefly described in Ch. \ref{CH:ac6_study}, are remnants of microbursts in the drift loss cone. One way to test this hypothesis is to look for the occurrence rates of curtains eastward and westward of the SAA. If curtains are electrons in the drift loss cone then the SAA will remove curtains as they drift to the east. Thus under the proposed hypothesis the number of curtains should be greater just to the west of the SAA than to the east. An alternative approach to test this hypothesis is to estimate how each curtain's flux changes between the two AC6 units. If curtains are drifting and have a falling energy spectra, then the larger number of slower-drifting, low energy, electrons will appear as an enhancement in the flux for the trailing spacecraft. If such a trend is apparent then curtains must be drifting, otherwise they may be actively scattered in the same location. 

\section{Future missions}
A few upcoming missions are dedicated to study microbursts and would be able to address some of the unknown questions raised in this dissertation and discussed below.

One approach to determine if microburst scattering is a diffusive or a non-linear process can be done in LEO where the transport of microburst electrons inside the loss cone can be more easily observed. In contrast to particle measurements made near the magnetic equator where the local loss cone is only a few degrees, the loss cone in LEO is $\approx 60^\circ$ which is much easier to resolve with an instrument with multiple look directions. With this measurement, different scattering mechanisms can be studied. If the scattering process is diffusive, then the microburst flux will be monotonically decreasing (or flat) deeper into the loss cone. A non-linear scattering process, on the other hand, will have a more complex pitch angle vs flux profile e.g. a relative maximum at $0^\circ$, followed by decreasing flux towards the loss cone boundary. One mission that plans to make this measurement is The Relativistic Electron Atmospheric Loss (REAL) CubeSat. This CubeSat, planned to launch in 2021, will sample the inside and outside of the loss cone with a solid state detector with a five look directions.

As previously mentioned, the unknown microburst shape makes microburst size comparisons between balloons and satellites ambiguous. One of the most feasible ways to resolve this ambiguity is to image microburst precipitation in the upper atmosphere using a balloon. This imaging is possible because when microburst electrons impact the atmosphere, they scatter with Earth's atmosphere and generate bremsstrahlung X-rays. These X-rays have a relatively long mean free path at $\approx 35$ km balloon altitudes so a balloon-borne imager will predominately observe primary X-rays emitted directly from the microburst electrons. This idea is the basis for the upcoming Balloon Observations Of Microburst Scales (BOOMS) mission. BOOMS will fly a set of X-ray pinhole imagers containing a scintillator crystal (to convert from X-rays to visible light) and a grid of photomultiplier tubes (PMT) underneath to record the distribution of light. The distribution of light across the grid of PMTs, together with instrument modeling, can be used to convert between the PMT signal and the angular position for each observed X-ray. Over a longer exposure, a probabilistic image can then be constructed of the microburst X-ray source. Then the microburst shape, and any spatial correlations of trains of microbursts can be observed.