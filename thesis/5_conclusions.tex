\chapter{Conclusions and Future Work}\label{conclusions}
\textcolor{blue}{
\begin{enumerate}
\item Microburst scattering mechanism and relation to prior work
\item Microburst sizes and the elliptical shape, mention the curtain-microburst ambiguity.
\end{enumerate}
}

In this dissertation we have explored the microburst scattering mechanism directly in Chapter \ref{CH:mageis_microburst} and indirectly in Chapters \ref{CH:bouncing_packet} and \ref{CH:ac6_study}. In Chapter \ref{CH:mageis_microburst} we used numerous particle and wave instruments on the Van Allen Probes and found signatures of microbursts with the Magnetic Electron Ion Spectrometer. To these observations we applied the relativistic theory of wave-particle resonant diffusion and found that the motion of the microburst electrons was not along single-wave particle characteristics in momentum phase space, given the spacecraft position and orientation and most probable wave and plasma parameters. This result at first appears to contradict the belief that many members of the community hold, that microburst precipitation is due to a diffusive process. In reality both are probably valid, just on different time scales. Individual microbursts are probably not scattered diffusively, but the combined contribution of an ensemble of microbursts will have properties that are well modeled as a diffusion process.

The microburst sizes estimated in prior literature as well as Chapters \ref{CH:bouncing_packet} and \ref{CH:ac6_study} show that there is a large variability in microburst sizes. The AC6 study in Chapter \ref{CH:ac6_study} showed that in LEO, most microbursts were observed while the AC6 separation was less than a few tens of km while a minority of microbursts were observed up to $\approx 100$ km separation. These conclusions agree with prior literature from high altitude balloons and LEO spacecraft. One unanswered question is what shape is a microburst? A circular microburst is easy to interpret and model due to its symmetry, but nature is not likely to be so perfect. A circular microburst near the scattering region will be deformed into an ellipse by the changing topology of Earth's magnetic field lines. One feasible solutions exists: a X-ray imager on a high altitude balloon which will be discussed in the next section.

\section{Future Work}
\textcolor{blue}{
\begin{itemize}
\item In future work mention how curtains should be studied and outline that project. Mention the SAA and amplitude idea.
\item Bouncing packet idea
\item Mention inverse microburst analysis and compare to Saito.
\item Mention BOOMS and how it plans to image microbursts to determine their shape without ambiguity. 
\end{itemize}
}

An extension of the case study in Chapter \ref{CH:mageis_microburst} will be a statistical study using the Van Allen Probes. Other microburst-like events have already been identified by eye. These other events are also simultaneously observed with enhanced wave activity, hence they may be related and a further investigation is warranted. A microburst detection scheme similar to the one used in Chapter \ref{CH:ac6_study} can be easily implemented to automatically identify other microbursts for further study. A few compelling questions that can be addressed with this study are: what is the typical pitch angle extent of microbursts? Do these microbursts have a similar MLT extent to microbursts observed in LEO? What fraction of microbursts were observed during enhanced wave activity? What wave modes and properties are observed during these events? And lastly, what fraction of microbursts can be modeled with a diffusive process?

Another approach to determine if microburst scattering is a diffusive or a non-linear process can be done in LEO. In contrast to particle measurements made near the magnetic equator where the local loss cone is only a few degrees, the loss cone in LEO is $\approx 60^\circ$ which is much easier to resolve with an instrument with multiple look directions. With this measurement, different scattering mechanisms can be discriminated. If the scattering process is diffusive, then the microburst flux will be monotonically decreasing (or flat) deeper into the loss cone. A non-linear scattering process, on the other hand, will have a more complex pitch angle vs flux profile e.g. a relative maximum at $0^\circ$, followed by decreasing flux towards the loss cone boundary. One mission that plans to make this measurement is The Relativistic Electron Atmospheric Loss (REAL) CubeSat. This CubeSat, planned to launch in 2021, will sample the inside the outside the loss cone with a solid state detector with a five look direction collimator.

As previously mentioned, the microburst shape is an unknown parameter that adds ambiguity when comparing the results from the AC6 study in Chapter \ref{CH:ac6_study} and prior literature from balloons. Imaging microburst precipitation is one of the most feasible ways to see the microburst shape. This imaging is possible because when microburst electrons impact the atmosphere, they scatter with Earth's atmosphere and generate bremsstrahlung X-rays. Then these X-rays have a relatively long mean free path above the Pfotzer maximum where a balloon-borne imager will predominately observe primary X-rays emitted directly from the microburst electrons. This idea is the basis for the upcoming Balloon Observations Of Microburst Scales (BOOMS) mission. The idea of BOOMS is to fly a set of X-ray pinhole imagers containing a scintillator crystal (to convert from X-rays to visible light) and a grid of photomultiplier tubes (PMT) underneath to record the distribution of light. With triangulation, this distribution of light across the grid of PMTs and instrument modeling can then be used to convert from the PMT signals to a the angular position of the X-ray. When exposed a longer duration, a probabilistic image can then be reconstructed of the X-ray source.

\textcolor{red}{With these results, it is always important to keep our methods and their biases in mind. Werner Heisenberg once wrote ``What we observe is not nature itself, but nature exposed to our method of questioning." \citet{Heisenberg1958}}