\chapter{Appendix C}\label{appendixc}
This appendix contains texts S1-S3. Text S1 derives the analytic model that transforms a prescribed microburst PDF into a $\bar{F}$ curve as a function of AC6 separation, $s$. Text S2 expands on the two-sized microburst model results presented in Section 5.3 and the range of optimal model parameters assuming continuous microburst PDFs such as the log-normal, Weibull, and Maxwellian. Lastly, text S3 presents the percent of microbursts observed in each separation bin, as a function of separation and compares is to the observed scale size of chorus waves as a function of wave amplitude.

\clearpage

%Delete all unused file types below. Copy/paste for multiples of each file type as needed.
\noindent\textbf{Text S1: Analytic Derivation of $\bar{F}(s)$}
Here we derive the integral form of $\bar{F}(s)$ under the following assumptions:

\begin{enumerate}
\item microbursts are circular with radius $r$
\item microbursts are randomly and uniformly distributed around AC6.
\end{enumerate}

First recall the area $A(r, s)$, given in Eq. 4 in the main text and copied here for convenience
\begin{equation}
A(r, s) = 2r^2 \cos^{-1}{\Big( \frac{s}{2r} \Big)} - \frac{s}{2} \sqrt{4r^2 - s^2}.
\end{equation} A circular microburst who's center lies in $A(r, s)$ will be observed by both AC6 units and is counted in $\bar{F}(s)$. Now we derive the integral form of $\bar{F}(s)$ that accounts for the different spacecraft separations and microburst sizes that are distributed by a hypothesized PDF $p(r, \theta)$.

First we will account for the effects of various spacecraft separation, assuming all microbursts are one size. For reference choose of radius, $r_0$ and spacecraft separation, $s_0$ such that $A(r_0, s_0) > 0$ which implies that some number of microbursts, $n_0$ will be simultaneously observed. Now, if for example the spacecraft separation (or microburst radius) is changed such that the area doubles, the second assumption implies that the number of microbursts observed during the same time interval must double as well. This can be expressed as 

\begin{equation} \label{density_eq}
\frac{n_0}{A(r_0, s_0)} = \frac{n}{A(r, s)}
\end{equation} and interpreted as the conservation of the microburst area density. By rewriting Eq. \ref{density_eq} as

\begin{equation}
n(r, s) = \bigg( \frac{n_0}{A(r_0, s_0)} \bigg) A(r, s)
\end{equation} it is more clear that the number of microbursts of size $r$ observed at separation $s$ is just $A(r, s)$ scaled by the reference microburst area density. The cumulative number of microbursts observed above $s$ is then

\begin{equation}
N(r, s) = \int_{s}^\infty n(r, s') ds' = \bigg( \frac{n_0}{A(r_0, s_0)} \bigg) \int_{s}^\infty A(r, s') ds'.
\end{equation} Lastly, $\bar{F}(s)$ for a single $r$ is then

\begin{equation}
\bar{F}(s) = \frac{N(s)}{N(0)} = \frac{\int_{s}^\infty A(r, s') ds'}{\int_{0}^\infty A(r, s') ds'}
\end{equation}

To incorporate a continuous microburst PDF such as $p(r) = p_1 \delta (r-r_1) + p_2 \delta (r-r_2) + ...$ we sum up the weighted number of microbursts that each size contributes to $N(s)$ i.e.

\begin{equation}
N(s) = \bigg( \frac{n_0}{A(r_0, s_0)} \bigg) \bigg( \int_{s}^\infty p_1 A(r_1, s') ds' + \int_{s}^\infty p_2 A(r_2, s') ds' + ...\bigg)
\end{equation}

The last step is to convert the sum of Dirac Delta functions into a continuous PDF $p(r)$ after which 

\begin{equation}
N(s) = \bigg( \frac{n_0}{A(r_0, s_0)} \bigg) \displaystyle\int\displaylimits_{s}^{\infty} \displaystyle\int\displaylimits_0^{\infty} A(r, s') p(r) dr ds'.
\end{equation} With these considerations, $\bar{F}(s)$ is then given by 

\begin{equation} \label{analytic_integral}
\bar{F}(s, \theta) = \frac{\displaystyle\int\displaylimits_{s}^{\infty} \displaystyle\int\displaylimits_0^{\infty} A(r, s') p(r, \theta) dr ds'}{\displaystyle\int\displaylimits_{0}^{\infty} \displaystyle\int\displaylimits_0^{\infty} A(r, s') p(r, \theta) dr ds'}
\end{equation}
\clearpage 

\noindent\textbf{Text S2: Most probable parameter values for continuous microburst PDFs}

Besides the one and two-size microburst models described in the main text, continuous PDFs such as the log-normal, Weibull, and Maxwellian were fit and their optimal parameters presented here.

For the Maxwellian PDF, we assumed the following form

\begin{equation}
p(r | a) = \sqrt{\frac{2}{\pi}} \frac{r^2 e^{-r^2/(2a^2)}}{a^3}.
\end{equation} The range of $a$ consistent with the observed data was found to be between 0 and 35 km. Next, the log-normal distribution of the following form was used
\begin{equation}
p(r | \mu, \sigma) = \frac{1}{\sigma r \sqrt{2 \pi}} e^{\Big( -\big( ln(r) - ln(\mu) \big)^2/(2 \sigma^2) \Big)}
\end{equation} and the results are summarized in \ref{table_s1}. Lastly the Weibull distribution of the following form was tested
\begin{equation}
p(r | c, r_0, \lambda) = c \bigg(\frac{r-r_0}{\lambda}\bigg)^{c-1} exp \Bigg(- \bigg(\frac{r-r_0}{\lambda}\bigg)^{c} \Bigg).
\end{equation} for which the model parameters are summarized in Table \ref{table_s2}.

\begin{table}[h]
\caption{Range of log-normal model parameters consistent with the observed AC6  $\bar{F}(s)$}
\label{table_s1}
\centering
\begin{tabular}{|c|c|c|}
\hline 
percentile (\%) & $\mu$ & $\sigma$ \\ 
\hline 
2.5 & 1.8 & 0 \\ 
\hline 
50 & 21.8 & 0.4 \\ 
\hline 
97.5 & 52.0 & 1.1 \\ 
\hline 
\end{tabular} 
\end{table}

\begin{table}[h]
\caption{Range of Weibull model parameters consistent with the observed AC6  $\bar{F}(s)$}
\label{table_s2}
\centering
\begin{tabular}{|c|c|c|c|}
\hline 
percentile (\%) & c & $r_0$ & $\lambda$ \\ 
\hline 
2.5 & 0.6 & 1.3 & 2.7 \\ 
\hline 
50 & 5.5 & 26.2 & 32 \\ 
\hline 
97.5 & 19.3 & 72.5 & 72.2 \\ 
\hline 
\end{tabular} 
\end{table}

\clearpage 
\noindent\textbf{Text S3: Comparison of microburst to whistler mode chorus $\bar{F}(s)$}

\textcolor{red}{TBD}
