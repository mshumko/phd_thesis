\chapter{Introduction}\label{CH:introduction}
Above Earth's atmosphere are the Van Allen radiation belts, a complex and dynamic plasma environment that can effect our technology-driven society. These effects include: a higher radiation dose for astronauts and cosmonauts, higher chance of spacecraft failure due to single event upsets that can lead to catastrophic latchups, cumulative degradation of silicon (changing the silicon doping) from an extended radiation dose that can degrade a transition to the point where it no longer function as a switch, and the degradation of the ozone layer due to the chemical production of NOX and HOX. With these effects in mind, it is no surprise that the radiation belts have been extensively studied since their discovery in the 1960s.

A topic of interest in the space physics community is wave-particle intersections that as we will see later in the introduction, can accelerate particles and scatter them into the atmosphere.

The goal of this dissertation is to study the wave-particle mechanism that scatters microbursts into Earth's atmosphere. This goal will be achieved by first introducing single charged particle motion in electric and magnetic fields, the major particle populations and how they couple in the magnetosphere, and then describe the history and current state of the fields relating to microbursts and wave-particle scattering

\section{Charged Particle Motion in Electric and Magnetic Fields}\label{Intro:particle_motion}
Charged particle motion in the magnetosphere, in particular inner magnetosphere ($< 8$ Re), can be approximated to first order by Earth's dipole magnetic field, and various electric fields. Here we describe three types of periodic motion that are experienced by charged particles in a dipolar magnetic fields. These motions are gyration, bounce, and drift and each of the three motions have a corresponding conserved quantity i.e. an adiabatic invariant. 

Gyration is the first motion considered here and it arises when a charged particle is in a magnetic field with magnitude $B$. This motion is circular with a Larmor radius of 
\begin{equation}
r = \frac{m v_\perp}{|q| B}
\end{equation} where m is the mass, $v_\perp$ the velocity perpendicular to $B$, $q$ is the charge of the particle. This motion has a corresponding gyrofrequency given by 
\begin{equation}
\Omega = \frac{|q| B}{m}
\end{equation} in units of radians/second.

\section{Particle Populations and Their Interractions in the Magnetosphere}\label{ntro:particle_populations}

\section{Radiation Belts}\label{Intro:radiation_belt}
The Van Allen radiation belts were discovered by \citet{Allen1959} and \citet{Vernov1960} during the Cold War.

\subsection{Inner Magnetosphere Coordinates}\label{Intro:coords}


 No one had predicted the
existence of Earth’s radiation belts—
nested tori of energetic particles trapped
by the planet’s magnetic field

\subsection{Particle Acceleration}\label{Intro:acceleration}

\subsubsection{Adiabatic Heating}\label{Intro:adiabatic_heating}

\subsubsection{Wave Resonance Heating}\label{Intro:wave_heating}

\subsection{Particle Losses}\label{Intro:acceleration}

\subsubsection{Electromagnetic Ion Cyclotron Wave Driven}\label{Intro:emic_scattering}

\subsubsection{Whistler Mode Chorus Wave Driven}\label{Intro:chorus_scattering}

\section{Scope of Reserach}\label{Intro:scope}
This dissertation furthers our understanding of the microburst scattering mechanism and is organized into the following chapters. Chapter \textcolor{red}{X} will describe the spacecraft missions used to study microburst precipitation and wave-particle scattering. Then Ch. \textcolor{red}{Y} will describe a microburst scattering event observed by NASA's Van Allen Probes and the quasi-linear diffusion model that was developed. Next, Ch. \textcolor{red}{Z} will describe a bouncing packet microburst observation made by MSU's FIREBIRD-II mission where the microburst's lower bound longitudinal and latitudinal scale sizes were estimated. Chapter \textcolor{red}{ZZ} then expands the case study result from Ch. \textcolor{red}{Z} to a statistical study of microburst sizes and the microburst size models developed to interpret the data. Lastly, \textcolor{red}{ZZZ} will summarize the dissertation work and make concluding remarks about research to be done.

\iffalse %%%%%%%%%%%%%%%%%%%%% TEMPLATE %%%%%%%%%%%%%%%%%%%%%%%%%%%%%%%%%%%%
Welcome to the Montana State University electronic Thesis/Dissertation (ETD) \LaTeX{} template.  In this chapter various sections, subsections, and subsubsections are created and filled with random text).  In Ch.~\ref{CH:theory} methods to write equations and how to include figures and tables are explored. Conclusions are drawn in Ch.~\ref{conclusion}.


\section{Section}\label{Sect:test}
\lipsum[1] % Random text

\subsection{Subsection}\label{Sect:testsub}
\lipsum[2] % Random text

\subsubsection{Subsubsection}\label{Sect:testsubsub}
\lipsum[3] % Random text

\longsubsection{Subsection With a Very Very Very Very}{Very Very Very Very Very Very Long Title}\label{Sect:longsub}
For long subsection titles use the command \verb|\longsubsection{#1}{#2}|, where \#1 is the first line of the long title, and \#2 is the second line of the long title. You can also pass an optional argument to this command that puts a shorter title in the table of contents as shown by the subsection below.

\longsubsection[Subsection With a Very Long Title]{Subsection With a Very Long Title}{But Shortened in the Table Of Contents}\label{Sect:longsub2}
There are \textbf{not} similar commands for sections and subsubsections as these are not specified in the MSU style guide.  
\fi
